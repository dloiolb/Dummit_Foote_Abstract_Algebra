\documentclass[a4paper,10pt]{book}
\usepackage[utf8]{inputenc}
\usepackage[T1]{fontenc}
\usepackage[]{mdframed}
\usepackage{lipsum}
\usepackage[dvipsnames]{xcolor}
\usepackage{amsfonts} 
\usepackage{times}
\usepackage[shortlabels]{enumitem}
\usepackage{amsmath}
\usepackage{centernot}
\usepackage{amsthm,amssymb}
\usepackage{verbatim}
\usepackage{multicol}
\usepackage{titletoc}
\usepackage{graphicx}
\usepackage{minitoc}

\usepackage[T1]{fontenc}
\usepackage{stix}
\renewcommand\familydefault{\sfdefault}

\usepackage[titles]{tocloft}
\renewcommand{\cftdot}{} 

\usepackage{amsthm}
% \usepackage{thmtools}

\newtheorem{theorem}{Theorem}[section] % the main one
\theoremstyle{plain} % just in case the style had changed

% Define a theorem style that doesn't show parentheses
\newtheoremstyle{noparens}%
  {}{}               % Space above/below
  { }         % Body font
  {}                 % Indent amount
  {}                 % Theorem head font (no "Theorem" label)
  {}                 % Punctuation after theorem number
  { }                % Space after theorem head
  {\textbf{\thmnumber{#2}}~\textbf{\thmnote{#3}}.} 

\theoremstyle{noparens}
\newtheorem{genericthm}{}[section]  % No "Theorem" word

\makeatletter
% \newcommand{\thistheoremname}{}
\newenvironment{namedthm}[1]{%
  \renewcommand{\thistheoremname}{#1}%
  \def\@currentlabel{\arabic{section}.\arabic{genericthm}}%
  \begin{genericthm}[\thistheoremname]%
}{\end{genericthm}}
\makeatother

\theoremstyle{plain} % just in case the style had changed
\newcommand{\thistheoremname}{}
\newtheorem*{genericthm*}{\thistheoremname}
\newenvironment{namedthm*}[1]
	{\renewcommand{\thistheoremname}{#1}%
	\begin{genericthm*}}
	{\end{genericthm*}}


\usepackage[T1]{fontenc}
\usepackage{geometry}
% \usepackage{titlesec}
% \usepackage{titletoc}
\usepackage{blindtext} 

% \usepackage{tocloft}
\usepackage{tocloft}
  \setlength{\cftchapnumwidth}{2.25cm}
  \renewcommand{\cftchappresnum}{Chapter }

  % \usepackage[toc]{multitoc}
% \renewcommand*{\multicolumntoc}{2}
\setlength{\columnseprule}{0.5pt}
\renewcommand{\cftchapleader}{\:\:}
\renewcommand{\cftchapafterpnum}{\cftparfillskip}
% \renewcommand{\cftpnumalign}{l}
\renewcommand{\cftsecleader}{\:\:}
\renewcommand{\cftsecafterpnum}{\cftparfillskip}
\renewcommand{\cftpartleader}{\:\:}
\renewcommand{\cftpartafterpnum}{\cftparfillskip}

% \renewcommand{\cftpartfont}{\sffamily}
% \renewcommand{\cftchapfont}{\sffamily}
\renewcommand{\cftsecfont}{\sffamily}
% \renewcommand{\cftsubsecfont}{\sffamily}
% \renewcommand{\cftpartpagefont}{\sffamily}
% \renewcommand{\cftchappagefont}{\sffamily}
\renewcommand{\cftsecpagefont}{\sffamily}
% \renewcommand{\cftsubsecpagefont}{\sffamily}


%   % ---------- Chapter formatting ----------
% \setlength{\cftchapindent}{0pt}        % no indent
% \setlength{\cftchapnumwidth}{3em}      % enough space for "10"
% \renewcommand{\cftchapfont}{\normalfont}
% \renewcommand{\cftchappagefont}{\normalfont}

% % ---------- Section formatting ----------
% \setlength{\cftsecindent}{3em}         % indent to align with chapter title
% \setlength{\cftsecnumwidth}{4em}       % enough space for "10.1"
% \renewcommand{\cftsecfont}{\normalfont}
% \renewcommand{\cftsecpagefont}{\normalfont}

%   \titlecontents{part}%
% [0pt]{\sffamily\bfseries\large\protect\addvspace{15pt}\titlerule\addvspace{1.5ex}}%remove rule if you like
% {}{\partname~}
% {\hfill\contentspage}%replaced with {} if don't want page number for parts
% [\addvspace{0.7ex}\titlerule\addvspace{1.5ex}]%remove rule if you like

% \titlecontents{part}%
% [0pt]{}%remove rule if you like
% {}{\MakeUppercase{\partname}~}
% {\cftpartleader\contentspage}%replaced with {} if don't want page number for parts
% [\addvspace{2ex}]%remove rule if you like


% \setlength{\cftchapnumwidth}{0pt}
% \renewcommand{\cftchappresnum}{\chaptername\ }
% \renewcommand{\cftchapaftersnum}{\ \ \ \ }
% \renewcommand{\cftchapaftersnumb}{\newline}
\renewcommand{\cftchapdotsep}{\cftdotsep}
% \renewcommand{\cftpartpresnum}{Part \ }   % Prefix before part number
% \renewcommand{\cftpartaftersnum}{hi} % Space after the number
\renewcommand{\cftpartpresnum}{\hfill}
\renewcommand{\cftpartleader}{\hfill}

% \renewcommand{\thesubsection}{\Alph{subsection}.}
\makeatletter
\@addtoreset{subsection}{section} % resets subsection counter at each new section
\makeatother
% \usepackage{tocloft}
\renewcommand{\cftsecnumwidth}{1.7cm}

\usepackage{url}
\usepackage{hyperref}
\usepackage{geometry}
\geometry{a4paper}
\usepackage[english]{babel}
\title{ABSTRACT ALGEBRA\\
    \large DUMMIT, FOOTE\\
    \large Second Edition\\
	\large Notes + Exercises\\
}
\author{J.B.}
\date{\small July 2025
}

\begin{document}


\pagenumbering{roman}
\maketitle
% hello
{\sffamily
\tableofcontents
}

\clearpage\pagenumbering{arabic}
%\pagenumbering{arabic}

\setcounter{chapter}{0}
\chapter*{Preliminaries}
\addcontentsline{toc}{chapter}{Preliminaries}

% 0.1
\section{Basics}
Let $f:A\to B$.
\begin{enumerate}[(1)]
        \item $f$ is \textit{injective} or is an \textit{injection} if whenever $a_1\neq a_2$, then $f(a_1)\neq f(a_2)$.
        \item $f$ is \textit{surjective} or is an \textit{surjection} if for all $b\in B$ there is some $a\in A$ such that $f(a)=b$; i.e., the image of $f$ is all of $B$. 
        (The codomain of $f$ is $B$, while the range/image of $f$ is the subset $f(A):=\{b\in B:b=f(a), \text{ for some }a\in A\}$)
        \item $f$ is \textit{bijective} or is an \textit{bijection} if it is both injective and surjective.
        \item $f$ has a \textit{left inverse} if there is a function $g:B\to A$ such that $g\circ f:A\to A$ is the identity map on $A$; i.e., $(g\circ f)(a)=a$, for all $a\in A$.
        \item $f$ has a \textit{right inverse} if there is a function $h:B\to A$ such that $f\circ h:B\to B$ is the identity map on $B$; i.e., $(f\circ h)(b)=b$, for all $b\in B$.
    \end{enumerate}
\begin{namedthm*}{Proposition 1}
    Let $f:A\to B$.
    \begin{enumerate}[(1)]
        \item The map $f$ is injective iff $f$ has a left inverse.
        \item The map $f$ is surjective iff $f$ has a right inverse.
        \item The map $f$ is a bijection iff there exists $g:B\to A$ such that $f\circ g$ is the identity map on $B$ and $g\circ f$ is the identity map on $A$.
        (The map $g$ is necessarily unique and we say $g$ is the 2-sided \textit{inverse} of $f$)
        \item If $A$ and $B$ are finite sets with the same number of elements ($|A|=|B|$), then $f:A\to B$ is bijective iff $f$ is injective iff $f$ is surjective.
    \end{enumerate}
    \begin{proof}
    
    \begin{enumerate}[(1)]
        \item Suppose $f$ is injective. 
        Now, note that by definition of image of $f$, for all $c\in f(A)$, there exists $a\in A$ s.t. $c=f(a)$.
        Thus for all such $c$, we may define the function $g:f(A)\to A$ by $g(f(a))=g(c):=a$.
        Note that $g$ is well-defined as a function because each unique $c\in B$ corresponds to a unique $a\in A$ ($c_1=f(a_1)= f(a_2)=c_2$ implies $g(c_1)=a_1=a_2=g(c_2)$).
        We may extend $g$ to all of $B$ arbitrarily.
        On the other hand, suppose $f$ has a left inverse.
        Consider any $a_1,a_2\in A$ such that $f(a_1)=f(a_2)$.
        Then $a_1=g(f(a_1))=g(f(a_2))=a_2$.
        \item Suppose $f$ is surjective. 
        Then for any $b\in B$, there exists some $a\in A$ such that $f(a)=b$. 
        Thus it is well-defined to define the function $h:B\to A$ such that $h(b)=a$, and we have $f(h(b))=f(a)=b$.
        On the other hand, suppose $f$ has a right inverse.
        Consider any $b\in B$.
        Then $f(h(b))=b$, with $a=h(b)\in A$
        \item Suppose $f$ is a bijection.
        Then by (1) and (2), there exists a left inverse $g$ and a right inverse $h$.
        Fix any $b\in B$.
        Then by surjectivity of $f$, there exists $a\in A$ such that $b=f(a)$.
        But then $g(b)=g(f(a))=a=h(b)$, and $g\equiv h$ is the inverse of $f$.
        \item Bijective implies injective and surjective by definition.
        Now suppose $f$ is injective.
        Suppose that for all $a\in A$ there does not exist $b\in B$ whence $f(a)=b$.
        But by the pidgeonhole principle there must be (distinct) $a_1\neq a_2\in A$ that map to the same element in $B$; i.e., $f(a_1)=f(a_2)$, and this is a contradiction to the injectivity.
        On the other hand suppose $f$ is surjective.
        Suppose that there exists $a_1\neq a_2\in A$ but $f(a_1)=f(a_2)$.
        Again by the pidgeonhole principle there must be a $b\in B$ that is not mapped to, which is a contradiction.
    \end{enumerate}
    \end{proof}
    Let $A$ be a nonempty set.
    \begin{enumerate}[(1)]
        \item A \textit{binary relation} on a set $A$ is a subset $R$ of $A\times A$ and we write $a\sim b$ if $(a,b)\in R$.
        \item The relation $\sim$ on $A$ is said to be:
        \begin{enumerate}[(a)]
            \item \textit{reflexive} if $a\sim a$ for all $a\in A$,
            \item \textit{symmetric} if $a\sim b$ implies $b\sim a$ for all $a,b\in A$,
            \item \textit{transitive} if $a\sim b$ and $b\sim c$ implies $a\sim c$ for all $a,b,c\in A$.
        \end{enumerate}
        A relation is an \textit{equivalence relation} if it is reflexive, symmetric, and transitive.
        \item If $\sim$ defines an equivalence relation on $A$, then the \textit{equivalence class} of $a\in A$ is defined to be $\{x\in A\mid x\sim a\}$.
        Elements of the equivalence class of $a$ are said to be \textit{equivalent} to $a$.
        If $C$ is an equivalence class, any element of $C$ is called a \textit{representative} of the class $C$.
        \item A \textit{partition} of $A$ is any collection $\{A_i\mid i\in I\}$ of nonempty subsets of $A$ ($I$ some indexing set) such that 
        \begin{enumerate}[(a)]
            \item $A=\cup_{i\in I}A_i$, and 
            \item $A_i\cap A_j=\emptyset$, for all $i,j\in I$ with $i\neq j$.
        \end{enumerate}
    \end{enumerate}
    \begin{namedthm*}{Preposition 2}
        Let $A$ be a nonempty set.
        \begin{enumerate}[(1)]
            \item If $\sim$ defines an equivalence relation on $A$ then the set of equivalence classes of $\sim$ form a partition of $A$.
            \item If $\{A_i\mid i\in I\}$ is a partition of $A$ then there is an equivalence relation on $A$ whose equivalence classes are precisely the sets $A_i,i\in I$.
        \end{enumerate}
    \end{namedthm*}
\end{namedthm*}
\subsection*{\centering EXERCISES}
In exercises 1 to 4 let $\mathcal{A}$ be the set of $2\times2$ matrices with real number entries.
Recall that matrix multiplication is defined by
\[
    \begin{pmatrix}
        a&b\\c&d
    \end{pmatrix}
    \begin{pmatrix}
        p&q\\r&s
    \end{pmatrix}
    =
    \begin{pmatrix}
        ap+br&aq+bs\\cp+dr&cq+ds
    \end{pmatrix}.
\]
Let 
\[
    M=\begin{pmatrix}
        1&1\\0&1
    \end{pmatrix}
\]
and let 
\[
    \mathcal{B}:=\{X\in\mathcal{A}\mid MX=XM\}.
\]
\begin{enumerate}
    \item Determine which of the following elements of $\mathcal{A}$ lie in $\mathcal{B}$:
    \[
        \begin{pmatrix}
            1&1\\0&1
        \end{pmatrix},
        \begin{pmatrix}
            1&1\\1&1
        \end{pmatrix},
        \begin{pmatrix}
        0&0\\0&0
    \end{pmatrix},
    \begin{pmatrix}
        1&1\\1&0
    \end{pmatrix},
    \begin{pmatrix}
        1&0\\0&1
    \end{pmatrix},
    \begin{pmatrix}
        0&1\\1&0
    \end{pmatrix}
    \]
    \ \\The first is trivially yes.
    The second is no:
    \[
    \begin{pmatrix}
            1&1\\0&1
        \end{pmatrix}
        \begin{pmatrix}
            1&1\\1&1
        \end{pmatrix}
        =
        \begin{pmatrix}
            2&2\\1&1
        \end{pmatrix}
        \neq
        \begin{pmatrix}
            1&2\\1&2
        \end{pmatrix}
        =
        \begin{pmatrix}
            1&1\\1&1
        \end{pmatrix}
        \begin{pmatrix}
            1&1\\0&1
        \end{pmatrix}
    \]
    The third is trivially yes.
    The fourth is no:
    \[
    \begin{pmatrix}
            1&1\\0&1
        \end{pmatrix}
        \begin{pmatrix}
            1&1\\1&0
        \end{pmatrix}
        =
        \begin{pmatrix}
            2&1\\1&0
        \end{pmatrix}
        \neq
        \begin{pmatrix}
            1&2\\1&1
        \end{pmatrix}
        =
        \begin{pmatrix}
            1&1\\1&0
        \end{pmatrix}
        \begin{pmatrix}
            1&1\\0&1
        \end{pmatrix}
    \]
    The fifth is yes (identity).
    The sixth is no:
    \[
    \begin{pmatrix}
            1&1\\0&1
        \end{pmatrix}
        \begin{pmatrix}
            0&1\\1&0
        \end{pmatrix}
        =
        \begin{pmatrix}
            1&1\\1&0
        \end{pmatrix}
        \neq
        \begin{pmatrix}
            0&1\\1&1
        \end{pmatrix}
        =
        \begin{pmatrix}
            0&1\\1&0
        \end{pmatrix}
        \begin{pmatrix}
            1&1\\0&1
        \end{pmatrix}
    \]
    \item Prove that if $P,Q\in\mathcal{B}$, then $P+Q\in\mathcal{B}$.
    \item Prove that if $P,Q\in\mathcal{B}$, then $P\cdot Q\in\mathcal{B}$.
    \item Find conditions on $p,q,r,s$ which determine precisely when $\begin{pmatrix}
        p&q\\r&s
    \end{pmatrix}\in\mathcal{B}$.
    \item Determine whether the following functions $f$ are well-defined:
    \begin{enumerate}
        \item $f : \mathbb{Q} \to \mathbb{Z}$ defined by $f(a/b) = a$;
        \item $f : \mathbb{Q} \to \mathbb{Q}$ defined by $f(a/b) = a^2/b^2$;

    \end{enumerate}
    \item Determine whether the function $f : \mathbb{R}^+ \to \mathbb{Z}$ defined by mapping a real number $r$ to the first digit to the right of the decimal point in a decimal expansion of $r$ is well defined.
    
    \item Let $f: A \to B$ be a surjective map of sets. Prove that the relation
    \[
        a \sim b \iff f(a) = f(b)
    \]
    is an equivalence relation whose equivalence classes are the fibers of $f$.
\end{enumerate}

% 0.2
\section{Properties of the Integers}

% 0.3
\section{$\mathbb{Z}/n\mathbb{Z}$: The Integers Modulo $n$}
\part{GROUP THEORY}
\chapter{Introduction to Groups}

% 1.1
\section{Basic Axioms and Examples}
\subsection*{\centering EXERCISES}

% 1.2
\section{Dihedral Groups}
\subsection*{\centering EXERCISES}

% 1.3
\section{Symmetric Groups}
\subsection*{\centering EXERCISES}

% 1.4
\section{Matrix Groups}
\subsection*{\centering EXERCISES}

% 1.5
\section{The Quaternion Group}
\subsection*{\centering EXERCISES}

% 1.6
\section{Homomorphisms and Isomorphisms}
\subsection*{\centering EXERCISES}

% 1.7
\section{Group Actions}
\subsection*{\centering EXERCISES}

\chapter{Subgroups}

% 2.1
\section{Definition and Examples}
\subsection*{\centering EXERCISES}

% 2.2
\section{Centralizers and Normalizers, Stabilizers and Kernels}
\subsection*{\centering EXERCISES}

% 2.3
\section{Cyclic Groups and Cyclic Subgroups}
\subsection*{\centering EXERCISES}

% 2.4
\section{Subgroups Generated by Subsets of a Group}
\subsection*{\centering EXERCISES}

% 2.5
\section{The Lattice of Subgroups of a Group}
\subsection*{\centering EXERCISES}

\chapter{Quotient Groups and Homomorphisms}

% 3.1
\section{Definitions and Examples}
\subsection*{\centering EXERCISES}

% 3.2
\section{More on Cosets and Lagrange's Theorem}
\subsection*{\centering EXERCISES}

% 3.3
\section{The Isomorphism Theorems}
\subsection*{\centering EXERCISES}

% 3.4
\section{Composition Series and the Hölder Program}
\subsection*{\centering EXERCISES}

% 3.5
\section{Transpositions and the Alternating Group}
\subsection*{\centering EXERCISES}


\chapter{Group Actions}

% 4.1
\section{Group Actions and Permutation Representations}

% 4.2
\section{Groups Acting on Themselves by Left Multiplication—Cayley's Theorem}

% 4.3
\section{Groups Acting on Themselves by Conjugation—The Class Equation}

% 4.4
\section{Automorphisms}

% 4.5
\section{The Sylow Theorems}

% 4.6
\section{The Simplicity of $A_n$}

\chapter{Direct and Semidirect Products and Abelian Groups}

% 5.1
\section{Direct Products}

% 5.2
\section{The Fundamental Theorem of Finitely Generated Abelian Groups}

% 5.3
\section{Table of Groups of Small Order}

% 5.4
\section{Recognizing Direct Products}

% 5.5
\section{Semidirect Products}


\chapter{Further Topics in Group Theory}

% 6.1
\section{$p$-groups, Nilpotent Groups, and Solvable Groups}

% 6.2
\section{Applications in Groups of Medium Order}

% 6.3
\section{A Word on Free Groups}
\part{RING THEORY}
\chapter{Introduction to Rings}

% 7.1
\section{Basic Definitions and Examples}

% 7.2
\section{Examples: Polynomial Rings, Matrix Rings, and Group Rings}

% 7.3
\section{Ring Homomorphisms and Quotient Rings}

% 7.4
\section{Properties of Ideals}

% 7.5
\section{Rings of Fractions}

% 7.6
\section{The Chinese Remainder Theorem}

\chapter{Euclidean Domains, Principal Ideal Domains, and Unique Factorization Domains}

% 8.1
\section{Euclidean Domains}

% 8.2
\section{Principal Ideal Domains (P.I.D.s)}

% 8.3
\section{Unique Factorization Domains (U.F.D.s)}


\chapter{Polynomial Rings}

% 9.1
\section{Definitions and Basic Properties}

% 9.2
\section{Polynomial Rings over Fields I}

% 9.3
\section{Polynomial Rings that are Unique Factorization Domains}

% 9.4
\section{Irreducibility Criteria}

% 9.5
\section{Polynomial Rings over Fields II}
\part{MODULES AND VECTOR SPACES}
\chapter{Introduction to Module Theory}

% 10.1
\section{Basic Definitions and Examples}

% 10.2
\section{Quotient Modules and Module Homomorphisms}

% 10.3
\section{Generation of Modules, Direct Sums, and Free Modules}

% 10.4
\section{Tensor Products of Modules}

% 10.5
\section{Exact Sequences—Projective, Injective, and Flat Modules}

\chapter{Vector Spaces}

% 11.1
\section{Definitions and Basic Theory}

% 11.2
\section{The Matrix of a Linear Transformation}

% 11.3
\section{Dual Vector Spaces}

% 11.4
\section{Determinants}

% 11.5
\section{Tensor Algebras, Symmetric and Exterior Algebras}


\chapter{Modules over Principal Ideal Domains}

% 12.1
\section{The Basic Theory}

% 12.2
\section{The Rational Canonical Form}

% 12.3
\section{The Jordan Canonical Form}
\part{FIELD THEORY AND GALOIS THEORY}
\chapter{Field Theory}

% 13.1
\section{Basic Theory of Field Extensions}

% 13.2
\section{Algebraic Extensions}

% 13.3
\section{Classical Straightedge and Compass Constructions}

% 13.4
\section{Splitting Fields and Algebraic Closures}

% 13.5
\section{Separable and Inseparable Extensions}

% 13.6
\section{Cyclotomic Polynomials and Extensions}


\chapter{Galois Theory}

% 14.1
\section{Basic Definitions}

% 14.2
\section{The Fundamental Theorem of Galois Theory}

% 14.3
\section{Finite Fields}

% 14.4
\section{Composite Extensions and Simple Extensions}

% 14.5
\section{Cyclotomic Extensions and Abelian Extensions over $\mathbb{Q}$}

% 14.6
\section{Galois Groups of Polynomials}

% 14.7
\section{Solvable and Radical Extensions: Insolvability of the Quintic}

% 14.8
\section{Computation of Galois Groups over $\mathbb{Q}$}

% 14.9
\section{Transcendental Extensions, Inseparable Extensions, Infinite Galois Groups}
\part{AN INTRODUCTION TO COMMUTATIVE RINGS, ALGEBRAIC GEOMETRY, AND HOMOLOGICAL ALGEBRA}
\chapter{Commutative Rings and Algebraic Geometry}

% 15.1
\section{Noetherian Rings and Affine Algebraic Sets}

% 15.2
\section{Radicals and Affine Varieties}

% 15.3
\section{Integral Extensions and Hilbert's Nullstellensatz}

% 15.4
\section{Localization}

% 15.5
\section{The Prime Spectrum of a Ring}

\chapter{Artinian Rings, Discrete Valuation Rings, and Dedekind Domains}

% 16.1
\section{Artinian Rings}

% 16.2
\section{Discrete Valuation Rings}

% 16.3
\section{Dedekind Domains}
\chapter{Introduction to Homological Algebra and Group Cohomology}

% 17.1
\section{Introduction to Homological Algebra—Ext and Tor}

% 17.2
\section{The Cohomology of Groups}

% 17.3
\section{Crossed Homomorphisms and $H^1(G, A)$}

% 17.4
\section{Group Extensions, Factor Sets, and $H^2(G, A)$}

\part{INTRODUCTION TO THE REPRESENTATION THEORY OF FINITE GROUPS}
\chapter{Representation Theory and Character Theory}

% 18.1
\section{Linear Actions and Modules over Group Rings}

% 18.2
\section{Wedderburn's Theorem and Some Consequences}

% 18.3
\section{Character Theory and the Orthogonality Relations}

\chapter{Examples and Applications of Character Theory}

% 19.1
\section{Characters of Groups of Small Order}

% 19.2
\section{Theorems of Burnside and Hall}

% 19.3
\section{Introduction to the Theory of Induced Characters}
\chapter*{Appendix I: Cartesian Products and Zorn's Lemma}
\addcontentsline{toc}{chapter}{Appendix I: Cartesian Products and Zorn's Lemma}
\chapter*{Appendix II: Category Theory}
\addcontentsline{toc}{chapter}{Appendix II: Category Theory}

\end{document}