\chapter*{Preliminaries}
\addcontentsline{toc}{chapter}{Preliminaries}

% 0.1
\section{Basics}
The set 
\[
    f(A)=\{b\in B\mid b\in f(a),\text{ for some }a\in A\},
\]
is a subset of $B$, called the \textit{range} or \textit{image} of $f$.
For each subset $C$ of $B$ the set 
\[
    f^{-1}(C)=\{a\in A\mid f(a)\in C\}
\]
consisting of the elements of $A$ mapping into $C$ under $f$ is called the \textit{preimage} or \textit{inverse image} of $C$ under $f$.
For each $b\in B$, the preimage of $\{b\}$ under $f$ is called the \textit{fiber} of $f$ over $b$.

Let $f:A\to B$.
\begin{enumerate}[(1)]
        \item $f$ is \textit{injective} or is an \textit{injection} if whenever $a_1\neq a_2$, then $f(a_1)\neq f(a_2)$.
        \item $f$ is \textit{surjective} or is an \textit{surjection} if for all $b\in B$ there is some $a\in A$ such that $f(a)=b$; i.e., the image of $f$ is all of $B$. 
        (The codomain of $f$ is $B$, while the range/image of $f$ is the subset $f(A):=\{b\in B:b=f(a), \text{ for some }a\in A\}$)
        \item $f$ is \textit{bijective} or is an \textit{bijection} if it is both injective and surjective.
        \item $f$ has a \textit{left inverse} if there is a function $g:B\to A$ such that $g\circ f:A\to A$ is the identity map on $A$; i.e., $(g\circ f)(a)=a$, for all $a\in A$.
        \item $f$ has a \textit{right inverse} if there is a function $h:B\to A$ such that $f\circ h:B\to B$ is the identity map on $B$; i.e., $(f\circ h)(b)=b$, for all $b\in B$.
    \end{enumerate}
\begin{namedthm*}{Proposition 1}
    Let $f:A\to B$.
    \begin{enumerate}[(1)]
        \item The map $f$ is injective iff $f$ has a left inverse.
        \item The map $f$ is surjective iff $f$ has a right inverse.
        \item The map $f$ is a bijection iff there exists $g:B\to A$ such that $f\circ g$ is the identity map on $B$ and $g\circ f$ is the identity map on $A$.
        (The map $g$ is necessarily unique and we say $g$ is the 2-sided \textit{inverse} of $f$)
        \item If $A$ and $B$ are finite sets with the same number of elements ($|A|=|B|$), then $f:A\to B$ is bijective iff $f$ is injective iff $f$ is surjective.
    \end{enumerate}
    \begin{proof}
    
    \begin{enumerate}[(1)]
        \item Suppose $f$ is injective. 
        Now, note that by definition of image of $f$, for all $c\in f(A)$, there exists $a\in A$ s.t. $c=f(a)$.
        Thus for all such $c$, we may define the function $g:f(A)\to A$ by $g(f(a))=g(c):=a$.
        Note that $g$ is well-defined as a function because each unique $c\in B$ corresponds to a unique $a\in A$ ($c_1=f(a_1)= f(a_2)=c_2$ implies $g(c_1)=a_1=a_2=g(c_2)$).
        We may extend $g$ to all of $B$ arbitrarily.
        On the other hand, suppose $f$ has a left inverse.
        Consider any $a_1,a_2\in A$ such that $f(a_1)=f(a_2)$.
        Then $a_1=g(f(a_1))=g(f(a_2))=a_2$.
        \item Suppose $f$ is surjective. 
        Then for any $b\in B$, there exists some $a\in A$ such that $f(a)=b$. 
        Thus it is well-defined to define the function $h:B\to A$ such that $h(b)=a$, and we have $f(h(b))=f(a)=b$.
        On the other hand, suppose $f$ has a right inverse.
        Consider any $b\in B$.
        Then $f(h(b))=b$, with $a=h(b)\in A$
        \item Suppose $f$ is a bijection.
        Then by (1) and (2), there exists a left inverse $g$ and a right inverse $h$.
        Fix any $b\in B$.
        Then by surjectivity of $f$, there exists $a\in A$ such that $b=f(a)$.
        But then $g(b)=g(f(a))=a=h(b)$, and $g\equiv h$ is the inverse of $f$.
        \item Bijective implies injective and surjective by definition.
        Now suppose $f$ is injective.
        Suppose that for all $a\in A$ there does not exist $b\in B$ whence $f(a)=b$.
        But by the pidgeonhole principle there must be (distinct) $a_1\neq a_2\in A$ that map to the same element in $B$; i.e., $f(a_1)=f(a_2)$, and this is a contradiction to the injectivity.
        On the other hand suppose $f$ is surjective.
        Suppose that there exists $a_1\neq a_2\in A$ but $f(a_1)=f(a_2)$.
        Again by the pidgeonhole principle there must be a $b\in B$ that is not mapped to, which is a contradiction.
    \end{enumerate}
    \end{proof}
    Let $A$ be a nonempty set.
    \begin{enumerate}[(1)]
        \item A \textit{binary relation} on a set $A$ is a subset $R$ of $A\times A$ and we write $a\sim b$ if $(a,b)\in R$.
        \item The relation $\sim$ on $A$ is said to be:
        \begin{enumerate}[(a)]
            \item \textit{reflexive} if $a\sim a$ for all $a\in A$,
            \item \textit{symmetric} if $a\sim b$ implies $b\sim a$ for all $a,b\in A$,
            \item \textit{transitive} if $a\sim b$ and $b\sim c$ implies $a\sim c$ for all $a,b,c\in A$.
        \end{enumerate}
        A relation is an \textit{equivalence relation} if it is reflexive, symmetric, and transitive.
        \item If $\sim$ defines an equivalence relation on $A$, then the \textit{equivalence class} of $a\in A$ is defined to be $\{x\in A\mid x\sim a\}$.
        Elements of the equivalence class of $a$ are said to be \textit{equivalent} to $a$.
        If $C$ is an equivalence class, any element of $C$ is called a \textit{representative} of the class $C$.
        \item A \textit{partition} of $A$ is any collection $\{A_i\mid i\in I\}$ of nonempty subsets of $A$ ($I$ some indexing set) such that 
        \begin{enumerate}[(a)]
            \item $A=\cup_{i\in I}A_i$, and 
            \item $A_i\cap A_j=\emptyset$, for all $i,j\in I$ with $i\neq j$.
        \end{enumerate}
    \end{enumerate}
    \begin{namedthm*}{Preposition 2}
        Let $A$ be a nonempty set.
        \begin{enumerate}[(1)]
            \item If $\sim$ defines an equivalence relation on $A$ then the set of equivalence classes of $\sim$ form a partition of $A$.
            \item If $\{A_i\mid i\in I\}$ is a partition of $A$ then there is an equivalence relation on $A$ whose equivalence classes are precisely the sets $A_i,i\in I$.
        \end{enumerate}
    \end{namedthm*}
\end{namedthm*}
\subsection*{\centering EXERCISES}
In exercises 1 to 4 let $\mathcal{A}$ be the set of $2\times2$ matrices with real number entries.
Recall that matrix multiplication is defined by
\[
    \begin{pmatrix}
        a&b\\c&d
    \end{pmatrix}
    \begin{pmatrix}
        p&q\\r&s
    \end{pmatrix}
    =
    \begin{pmatrix}
        ap+br&aq+bs\\cp+dr&cq+ds
    \end{pmatrix}.
\]
Let 
\[
    M=\begin{pmatrix}
        1&1\\0&1
    \end{pmatrix}
\]
and let 
\[
    \mathcal{B}:=\{X\in\mathcal{A}\mid MX=XM\}.
\]
\begin{enumerate}
    \item Determine which of the following elements of $\mathcal{A}$ lie in $\mathcal{B}$:
    \[
        \begin{pmatrix}
            1&1\\0&1
        \end{pmatrix},
        \begin{pmatrix}
            1&1\\1&1
        \end{pmatrix},
        \begin{pmatrix}
        0&0\\0&0
    \end{pmatrix},
    \begin{pmatrix}
        1&1\\1&0
    \end{pmatrix},
    \begin{pmatrix}
        1&0\\0&1
    \end{pmatrix},
    \begin{pmatrix}
        0&1\\1&0
    \end{pmatrix}
    \]
    \ \\The first is trivially yes.
    The second is no:
    \[
    \begin{pmatrix}
            1&1\\0&1
        \end{pmatrix}
        \begin{pmatrix}
            1&1\\1&1
        \end{pmatrix}
        =
        \begin{pmatrix}
            2&2\\1&1
        \end{pmatrix}
        \neq
        \begin{pmatrix}
            1&2\\1&2
        \end{pmatrix}
        =
        \begin{pmatrix}
            1&1\\1&1
        \end{pmatrix}
        \begin{pmatrix}
            1&1\\0&1
        \end{pmatrix}
    \]
    The third is trivially yes.
    The fourth is no:
    \[
    \begin{pmatrix}
            1&1\\0&1
        \end{pmatrix}
        \begin{pmatrix}
            1&1\\1&0
        \end{pmatrix}
        =
        \begin{pmatrix}
            2&1\\1&0
        \end{pmatrix}
        \neq
        \begin{pmatrix}
            1&2\\1&1
        \end{pmatrix}
        =
        \begin{pmatrix}
            1&1\\1&0
        \end{pmatrix}
        \begin{pmatrix}
            1&1\\0&1
        \end{pmatrix}
    \]
    The fifth is yes (identity).
    The sixth is no:
    \[
    \begin{pmatrix}
            1&1\\0&1
        \end{pmatrix}
        \begin{pmatrix}
            0&1\\1&0
        \end{pmatrix}
        =
        \begin{pmatrix}
            1&1\\1&0
        \end{pmatrix}
        \neq
        \begin{pmatrix}
            0&1\\1&1
        \end{pmatrix}
        =
        \begin{pmatrix}
            0&1\\1&0
        \end{pmatrix}
        \begin{pmatrix}
            1&1\\0&1
        \end{pmatrix}
    \]
    \item Prove that if $P,Q\in\mathcal{B}$, then $P+Q\in\mathcal{B}$.
%     \begin{align*}
%     \left(\begin{pmatrix}
%         a&b\\c&d
%     \end{pmatrix}
%     +
%     \begin{pmatrix}
%         p&q\\r&s
%     \end{pmatrix}
%     \right)
%     \begin{pmatrix}
%         1&1\\0&1
%     \end{pmatrix}
%     =\begin{pmatrix}
%         a&b\\c&d
%     \end{pmatrix}
%     \begin{pmatrix}
%         1&1\\0&1
%     \end{pmatrix}
%     +
%     \begin{pmatrix}
%         p&q\\r&s
%     \end{pmatrix}
%     \begin{pmatrix}
%         1&1\\0&1
%     \end{pmatrix}
%     \\=\begin{pmatrix}
%         1&1\\0&1
%     \end{pmatrix}
%     \begin{pmatrix}
%         a&b\\c&d
%     \end{pmatrix}
%     +
%     \begin{pmatrix}
%         1&1\\0&1
%     \end{pmatrix}
%     \begin{pmatrix}
%         p&q\\r&s
%     \end{pmatrix}
%     \\=\begin{pmatrix}
%         1&1\\0&1
%     \end{pmatrix}
%     \left(\begin{pmatrix}
%         a&b\\c&d
%     \end{pmatrix}
%     +
%     \begin{pmatrix}
%         p&q\\r&s
%     \end{pmatrix}
%     \right)
% \end{align*}
\[
        (P+Q)M=PM+QM=MP+MQ=M(P+Q)
\]
    \item Prove that if $P,Q\in\mathcal{B}$, then $P\cdot Q\in\mathcal{B}$.
    \[
        (PQ)M=P(QM)=P(MQ)=(PM)Q=(MP)Q=M(PQ)
\]
    \item Find conditions on $p,q,r,s$ which determine precisely when $\begin{pmatrix}
        p&q\\r&s
    \end{pmatrix}\in\mathcal{B}$.
    \[
    \begin{pmatrix}
        p&p+q\\r&r+s
    \end{pmatrix}=
    \begin{pmatrix}
        p&q\\r&s
    \end{pmatrix}
    \begin{pmatrix}
        1&1\\0&1
    \end{pmatrix}
    =
    \begin{pmatrix}
        1&1\\0&1
    \end{pmatrix}
    \begin{pmatrix}
        p&q\\r&s
    \end{pmatrix}
    =
    \begin{pmatrix}
        p+r&q+s\\r&s
    \end{pmatrix}
\]
Thus we have 
\begin{align*}
    \begin{cases}
        p=p+r\\
        r=r\\
        p+q=q+s\\
        r+s=s
    \end{cases}
    \implies
    \begin{cases}
        0=r\\
        p=s
    \end{cases}
\end{align*}
    \item Determine whether the following functions $f$ are well-defined:
    \begin{enumerate}
        \item $f : \mathbb{Q} \to \mathbb{Z}$ defined by $f(a/b) = a$;
        
        Yes, because the rational numbers are defined to be $\{a/b:a,b\in\mathbb{Z}, b\neq0\}$.
        \item $f : \mathbb{Q} \to \mathbb{Q}$ defined by $f(a/b) = a^2/b^2$;
        
        Yes, because $a,b\in\mathbb{Z}\implies a^2,b^2\in\mathbb{Z}$, and $b\neq0\implies b^2\neq0$.
    \end{enumerate}
    \item Determine whether the function $f : \mathbb{R}^+ \to \mathbb{Z}$ defined by mapping a real number $r$ to the first digit to the right of the decimal point in a decimal expansion of $r$ is well defined.
    
    False: see $0.0\overline{9}=0.1$, but $0=f(0.0\overline{9})=f(0.1)=1$, and $f$ is not a function.
    \item Let $f: A \to B$ be a surjective map of sets. Prove that the relation
    \[
        a \sim b \iff f(a) = f(b)
    \]
    is an equivalence relation whose equivalence classes are the fibers of $f$.

    See that $f(a)=f(a)$, and $f(a)=f(b)$ implies $f(b)=f(a)$, and $f(a)=f(b)$ and $f(b)=f(c)$ implies $f(a)=f(b)=f(c)$.
    Also see that
    \[
        f^{-1}(\{b\})=\{a\in A\mid f(a)=b\}.
    \]
\end{enumerate}

% 0.2
\section{Properties of the Integers}

% 0.3
\section{$\mathbb{Z}/n\mathbb{Z}$: The Integers Modulo $n$}